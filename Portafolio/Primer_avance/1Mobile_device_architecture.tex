\documentclass{article}
\title{Mobile device architecture}
\author{Higuera Sanchez Dulce Mariela}
\date{\today} 
\usepackage{multicol} 

\begin{document} 
	
	\maketitle This document describes the architecture of mobile devices. It mentions the characteristics of the various mobile operating systems, as well as the specific use of each operating system. The components of the devices are identified and distinguished to understand their structure and composition.
	
	\begin{multicols}{2}
		\section{Introduction}
		The architecture of mobile devices refers to the internal structure and design of portable electronic devices, such as smartphones, tablets, smartwatches, and wearable devices. 
		
		It consists of a combination of hardware and software that work together to provide specific functionalities and a smooth user experience.
		
		\section{Mobil devices}
		Before going into detail about the architecture, it is essential to understand what a mobile device is:
		
		A mobile device is defined as a portable electronic device designed to facilitate communication, access to information, and the performance of various tasks while the user is on the move. These devices are usually compact and lightweight, making them easy to carry around.
		
		\subsection{Mobile device features}
		Mobile devices are versatile tools that allow for a wide range of functions. They facilitate communication through phone calls, text messages, emails, and instant messaging applications, as well as access to the internet to browse websites, social networks, and cloud services. 
		
		Additionally, they offer multimedia functions such as taking photos, recording videos, and playing music and video. With the ability to install and run a variety of applications, mobile devices are useful for productivity, entertainment, education, and health activities. 
		
		They are also equipped with a variety of sensors, such as GPS, accelerometers, and gyroscopes, which enable additional functions such as navigation, motion detection, and augmented reality.
		
		\subsection{Mobile device components}
		Mobile device components can vary depending on the device, but generally include:
		
		• Processor (CPU): Responsible for executing instructions and processing data.
		
		• RAM: Temporarily stores data and running programs.
		
		• Internal storage: Where data and applications are permanently saved.
		
		• Touchscreen: Allows interaction with the device through touch gestures.
		
		• Battery: Supplies power for device operation.
		
		• Cameras: For capturing photos and videos.
		
		• Sensors: Such as GPS, accelerometer, etc., enabling various functions like navigation, motion detection, augmented reality, among others.
		
		• Wi-Fi connectivity: Allows connection to wireless networks for accessing the Internet and other devices.
		
		A relevant component is the gyroscope, a sensor that measures the angular velocity or angular orientation of a device in space. It works by detecting changes in the device's orientation and providing information about its rotational movement.
		
		The gyroscope consists of a disk or rotor that freely rotates around an axis. When the device moves or rotates, the gyroscope detects these changes in the direction of movement and converts them into electrical signals that can be interpreted by the device's operating system and applications.
		
		\section{Relating architecture and mobile devices}
		
		The relationship between a mobile device and its architecture allows us to understand how it functions, as well as how these devices are designed and optimized. The architecture of a mobile device refers to its internal structure, which includes both hardware and software.
		
		
		Some ways in which a mobile device relates to its architecture are:
		
		
		
	
		• Hardware: The hardware architecture of a mobile device includes physical components. The selection and arrangement of these components largely determine the device's performance, energy efficiency, and capabilities.
		
		
		• Software: The software architecture of a mobile device refers to the operating system as well as additional layers of software and applications running on the device. This includes the operating system kernel, device drivers, software libraries, user interfaces, and user applications.
		
		• Interaction between hardware and software: The architecture of a mobile device also encompasses how hardware and software interact with each other. The operating system manages hardware resources, how applications access and utilize these resources, and how performance and energy efficiency are optimized through coordination between hardware and software.
		
		• Evolution and improvements: The architecture of a mobile device is not static and evolves over time to adapt to new technologies, trends, and market needs. Therefore, the architecture must be flexible and scalable to allow for hardware and software updates, as well as improvements in performance, security, and usability of the device.
		
		\section{Operations systems}
		An operating system (OS) is fundamental software that acts as an intermediary between a computer's hardware and application programs. 
		
		It is a set of programs and utilities that manage system resources and provide services to application programs. Operating systems for mobile devices efficiently manage hardware resources such as CPU, RAM, and storage, as well as input/output devices like touchscreen and camera, allocating and releasing resources as needed to ensure optimal performance. 
		
		They provide an intuitive user interface, whether through a touchscreen with icons and gestures or via voice commands, allowing users to interact easily with their devices.
		
		\subsection{Operations systems}
		Mobile operating systems have distinctive features that make them unique and adaptable to different users' needs and preferences.
		
		• Android: Android is known for its wide variety of devices and customization. It offers a vast range of applications through the Google Play Store and allows deep customization of the operating system.
		Specific use: Android is popular among a wide range of users due to its variety of devices, price points, and customization. It is used by manufacturers like Samsung, Huawei, Xiaomi, and others.
		
		
		•iOS (Apple):
		It is simple and provides consistency in the user experience. It offers an integrated ecosystem with Apple devices such as iPhone, iPad, Apple Watch, and Mac, enabling a seamless experience across devices.
		Specific use: iOS is popular among users who prefer simplicity and integration across devices. It is commonly used by those who are part of the Apple ecosystem and value the quality of hardware and software.
		
		
		•Windows Mobile:
		It offers integration with Windows on computers and a familiar tile-based user interface. It focuses on productivity with Office applications and Microsoft services.
		Specific use: Although it has lost relevance in recent years, Windows Mobile is used by some business users who value integration with the Windows ecosystem and Microsoft's productivity tools.
		
		There are others like KaiOS, used in basic devices and low-end smartphones, and Linux-based operating systems like Ubuntu Touch and Sailfish OS, which offer an alternative for more technical users and privacy enthusiasts.
		
		\section{Conclusion}
		The architecture of mobile devices encompasses both the hardware and software that make up these portable devices. From the processor and memory to the operating system and applications, each component contributes to the functionality and performance of the device.
		
		The combination of operating system and hardware components influences the user experience and determines the capabilities and performance of the mobile device.
		
		
	\end{multicols} 
	\section{References}
	Vujevic, T. (2023, march). "The mobile app architecture guide". Decode Agency. https://decode.agency/article/mobile-app-architecture/
	
	
	
	Shah, M. (2023, October 18th). "Mobile App Architecture: Everything You Need to Know". Radix.
	https://radixweb.com/blog/guide-to-mobile-app-architecture
	
	
	
	Walburg, M. (2021, January 12th). "Mobile architecture – what are the types?". Binarapps
	https://binarapps.com/mobile-architecture-what-are-the-types/
	
\end{document} 
