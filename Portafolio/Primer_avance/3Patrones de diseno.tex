\documentclass{article}
\title{PATRONES DE DISEÑO}
\author{Higuera Sanchez Dulce Mariela}
\date{Febrero 3 de 2024} 
\usepackage{multicol} 

\begin{document} 
	
	\maketitle 
	El presente documento describe en que consisten los patrones de diseño, asi como sus carácteristicas y su relación con los dispositivos moviles.
	
	\begin{multicols}{2}
		
		\section{Introduccion}
		Los patrones de diseño son soluciones generales y reutilizables a problemas comunes que se encuentran al diseñar software. Estos patrones proporcionan un enfoque estructurado para resolver problemas de diseño y ayudan a los desarrolladores a crear código más flexible, mantenible y escalable. Los patrones de diseño capturan las mejores prácticas y experiencias previas en el desarrollo de software, ofreciendo una guía para abordar problemas específicos de manera eficiente y efectiva.
		
		\section{Carácteristicas de los patrones de diseño}
		section{Características}
		
		\textbf{Reutilización:} Los patrones de diseño son soluciones probadas y comprobadas para problemas recurrentes en el diseño de software, lo que permite su reutilización en diferentes contextos.
		
		\textbf{Flexibilidad:} Los patrones de diseño proporcionan una estructura flexible que puede adaptarse y aplicarse a diferentes situaciones y requisitos de diseño.
		
		\textbf{Abstracción:} Los patrones de diseño abstraen los detalles específicos de implementación, centrándose en conceptos y principios generales que pueden aplicarse de manera más amplia.
		
		\textbf{Claridad y comunicación:} Facilitan la comunicación entre los miembros del equipo al proporcionar un vocabulario común y una comprensión compartida de las soluciones de diseño.
		
		\textbf{Documentación:} Los patrones de diseño suelen estar bien documentados, lo que facilita su comprensión y aplicación por parte de los desarrolladores.
		
		\textbf{Estandarización:} Promueven la estandarización y las mejores prácticas en el diseño de software al ofrecer soluciones probadas y recomendadas para problemas comunes.
		
		\textbf{Evolución:} Los patrones de diseño evolucionan con el tiempo para adaptarse a las cambiantes necesidades y tecnologías del desarrollo de software, incorporando nuevas ideas y enfoques a medida que surgen.
		
		\section{Utilidad de los patrones de diseño}
		Los patrones de diseño son herramientas fundamentales en el desarrollo de software, ya que proporcionan un marco estructurado para abordar problemas de diseño específicos. Facilitan la creación de sistemas modulares y reutilizables al promover la separación de preocupaciones y la creación de componentes independientes. 
		
		Además, ayudan a evitar errores comunes al encapsular soluciones probadas para problemas recurrentes y siguen las mejores prácticas establecidas por la comunidad de desarrollo. Estos patrones también promueven la extensibilidad y la flexibilidad, permitiendo que los sistemas evolucionen y se adapten a nuevos requisitos y cambios en el entorno. 
		
		Además, facilitan la comunicación entre los miembros del equipo de desarrollo al utilizar un lenguaje común y una terminología compartida. 
		
	\end{multicols} 
	\section{References}
	
	Soto, N. (2021, July). "¿QUÉ SON LOS PATRONES DE DISEÑO?". Craft code. 
	https://craft-code.com/que-son-los-patrones-de-diseno/
	
	
	Hilman, G. (2023).Design Patterns. Refactoring Guru.
	https://refactoring.guru/design-patterns
	
\end{document} 
