\documentclass{article}
\title{TIPOS DE APLICACIONES}
\author{Higuera Sanchez Dulce Mariela}
\date{Febrero 03 de 2024} 
\usepackage{multicol} 

\begin{document} 
	
	\maketitle El presente documento describe los tipos de aplicaciones que existen,  menciona las ventajas y desventajas de cada una, así como los sistemas operativos en los que se suelen implementar.
	
	\begin{multicols}{2}
		\section{Introducción}
		Una aplicación se define como un programa informático diseñado para realizar una función específica o una serie de funciones para el usuario. Estas funciones pueden variar ampliamente, desde realizar cálculos matemáticos simples hasta gestionar tareas complejas como el correo electrónico, la edición de fotos o la navegación por internet. Las aplicaciones pueden ejecutarse en una variedad de dispositivos, como computadoras, teléfonos inteligentes, tabletas, relojes inteligentes y otros dispositivos electrónicos, y pueden ser desarrolladas para diferentes sistemas operativos como iOS, Android, Windows, entre otros.
		
		
		\section{Aplicaciones nativas}
		
		Las aplicaciones nativas están diseñadas específicamente para una plataforma en particular, como iOS (para dispositivos Apple) o Android (para dispositivos Android). Estas aplicaciones se desarrollan utilizando los lenguajes de programación y las herramientas de desarrollo recomendadas por el fabricante de la plataforma.
		
		
		\subsection{Ventajas de las apps nativas}
		
		• Rendimiento optimizado: Están diseñadas específicamente para una plataforma, lo que permite un rendimiento rápido y eficiente.
		
		• Acceso completo a las características del dispositivo: Pueden aprovechar todas las capacidades del hardware y del sistema operativo.
		
		• Experiencia de usuario superior: Al estar optimizadas para una plataforma específica, ofrecen una experiencia más intuitiva y fluida.
		
		• Mejor integración con el ecosistema: Pueden integrarse perfectamente con otras aplicaciones y servicios del sistema operativo.
		
		
		\subsection{Desventajas de las apps nativas }
		
		• Costo y tiempo de desarrollo: El desarrollo de aplicaciones nativas para diferentes plataformas puede ser costoso y llevar mucho tiempo, ya que requiere la creación de versiones separadas para cada sistema operativo.
		
		• Alcance limitado: Debido a que están diseñadas para una plataforma específica, las aplicaciones nativas pueden tener un alcance limitado y solo pueden ser utilizadas por usuarios que tengan dispositivos compatibles con ese sistema operativo.
	
		• Mantenimiento complicado: Mantener varias versiones de una aplicación nativa puede ser complicado, ya que cualquier actualización o corrección de errores debe implementarse por separado en cada plataforma, lo que aumenta la carga de trabajo para los desarrolladores.
		
		• Curva de aprendizaje: Los desarrolladores necesitan conocimientos específicos de cada plataforma para crear aplicaciones nativas, lo que puede requerir una curva de aprendizaje pronunciada y dificultar la gestión de equipos multidisciplinarios.
		
	
		\section{Aplicaciones no nativas}
		También conocidas como aplicaciones híbridas, son programas informáticos desarrollados utilizando tecnologías web estándar como HTML, CSS y JavaScript. A diferencia de las aplicaciones nativas, que están diseñadas específicamente para una plataforma en particular (como iOS o Android), las aplicaciones no nativas están diseñadas para ser ejecutadas en múltiples plataformas.
		
		Estas aplicaciones suelen ser desarrolladas utilizando frameworks como Apache Cordova, Ionic o React Native, que permiten escribir una vez el código y luego empaquetarlo como una aplicación nativa para cada plataforma específica.
		
		
		\subsection{Ventajas de las apps no nativas}
		• Desarrollo más rápido: Se pueden desarrollar más rápidamente al compartir gran parte del código base entre plataformas.
		
		• Menor costo: El desarrollo y mantenimiento de una sola aplicación para múltiples plataformas puede ser más económico que desarrollar aplicaciones nativas separadas
		
		• Flexibilidad: Pueden ejecutarse en múltiples plataformas, lo que permite alcanzar a una audiencia más amplia.
		
		• Fácil actualización: Los cambios y actualizaciones en el código base se reflejan automáticamente en todas las plataformas.

		\subsection{Desventajas de las apps no nativas}
    	• Rendimiento inferior: Las aplicaciones no nativas pueden experimentar un rendimiento inferior en comparación con las aplicaciones nativas, debido a la dependencia de tecnologías web estándar y la necesidad de ejecutar un contenedor web para su funcionamiento.

		• Limitaciones de acceso a funciones del dispositivo: A veces, las aplicaciones no nativas pueden tener dificultades para acceder a ciertas funciones del dispositivo, como sensores, cámaras o hardware específico, lo que puede limitar su funcionalidad en comparación con las aplicaciones nativas.

		• Experiencia de usuario menos óptima: Debido a su naturaleza híbrida, las aplicaciones no nativas pueden ofrecer una experiencia de usuario menos fluida y coherente en comparación con las aplicaciones nativas, especialmente en términos de rendimiento y respuesta.

		• Dependencia de actualizaciones de terceros: Las aplicaciones no nativas a menudo dependen de frameworks y bibliotecas de terceros para su funcionamiento, lo que puede generar problemas de compatibilidad y seguridad si estas herramientas no se actualizan regularmente o son abandonadas por sus desarrolladores.
		
		\section{Aplicaciones multiplataforma}
		Las aplicaciones multiplataforma son programas informáticos diseñados para ser ejecutados en múltiples plataformas, como iOS, Android, Windows, entre otras. Estas aplicaciones se desarrollan utilizando un único conjunto de tecnologías y herramientas, lo que permite que el mismo código base pueda ser utilizado en diferentes sistemas operativos.
	
		Para desarrollar aplicaciones multiplataforma, se utilizan frameworks o plataformas de desarrollo que facilitan la escritura de código una vez y su ejecución en diferentes sistemas operativos. 
		
		
		\subsection{Ventajas apps multiplataforma}
		Código base compartido: Permite escribir y mantener un solo conjunto de código para múltiples plataformas, lo que reduce el tiempo y los costos de desarrollo.
		1. Amplio alcance: Pueden llegar a una audiencia más amplia al ejecutarse en múltiples sistemas operativos.
		2. Herramientas y frameworks disponibles: Existen numerosos frameworks y herramientas disponibles para el desarrollo multiplataforma, lo que facilita el proceso de desarrollo.
		3. Mantenimiento simplificado: Las actualizaciones y correcciones de errores se pueden implementar de manera más eficiente en todas las plataformas.
		
		\subsection{Desventajas apps multiplataforma}
		
		1. Rendimiento y optimización: Debido a la necesidad de adaptarse a diferentes sistemas operativos, las aplicaciones multiplataforma pueden experimentar un rendimiento inferior y una optimización reducida en comparación con las aplicaciones nativas, lo que puede afectar la experiencia del usuario.
		2. Limitaciones de acceso a funciones del dispositivo: A menudo, las aplicaciones multiplataforma pueden tener dificultades para acceder a ciertas funciones del dispositivo o aprovechar plenamente el hardware específico, lo que puede limitar su funcionalidad en comparación con las aplicaciones nativas.
		3. Compromisos en el diseño y la experiencia del usuario: La necesidad de mantener la coherencia visual y funcional entre diferentes plataformas puede llevar a compromisos en el diseño y la experiencia del usuario, lo que puede resultar en una aplicación menos intuitiva y atractiva.
		4. Complejidad de desarrollo y mantenimiento: Si bien el desarrollo multiplataforma puede reducir el tiempo y los costos de desarrollo al compartir código base, también puede introducir una mayor complejidad en el proceso de desarrollo y mantenimiento, especialmente a medida que la aplicación crece en tamaño y complejidad.
		
		\section{Conclusion}
		Las aplicaciones nativas pueden requerir un mayor tiempo y costo de desarrollo al necesitar versiones separadas para cada plataforma, las no nativas pueden experimentar un rendimiento inferior y limitaciones en el acceso a ciertas funciones del dispositivo,mientras que las aplicaciones multiplataforma pueden comprometer el rendimiento y la experiencia del usuario, ya que deben adaptarse a diferentes sistemas operativos, lo que podría resultar en una menor optimización y funcionalidad.
		
		
	\end{multicols} 
	\section{References}
	Gunka, S. (2016, september). "Apps nativas: ¿Qué son y qué ventajas ofrecen? Studio G. https://gunkastudios.com/apps-nativas-que-son-y-que-ventajas-ofrecen/
	
	
	Gillis, A. (2023, October 18th). "Definition
	native app ". TechTarget
	https://www.techtarget.com/searchsoftwarequality/definition/native-application-native-app
	
	
	Kotlin, E. (2024, January 22th). "What is cross-platf	orm mobile development?". 
	https://kotlinlang.org/docs/cross-platform-mobile-development.html/
	
\end{document} 
