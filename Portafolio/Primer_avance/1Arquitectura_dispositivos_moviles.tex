\documentclass{article}
\title{ARQUITECTURA DE DISPOSITIVOS MOVILES}
\author{Higuera Sanchez Dulce Mariela}
\date{Febrero 03 de 2024} 
\usepackage{multicol} 

\begin{document} 
	
	\maketitle El presente documento describe en que consiste la arquitectura de dispositivos moviles,  menciona las caractericas de los diversos sistemas operativos para los moviles, asi como el uso particular de cada sistema operativo, se encuentran y distinguen los componentes de los dispositivos para comprender su estructura y composición.
	
	\begin{multicols}{2}
		\section{Introducción}
		La arquitectura de los dispositivos móviles se refiere a la estructura interna y el diseño de dispositivos electrónicos portátiles, como teléfonos inteligentes, tabletas, relojes inteligentes y dispositivos portátiles.
		
		\section{Dispositivos moviles}
Antes de entrar en detalle de la arquitectura, es indispensable conocer que es un dispositivo movil: Se define como un dispositivo electrónico portátil diseñado para facilitar la comunicación, el acceso a la información y la realización de diversas tareas mientras el usuario está en movimiento. Estos dispositivos suelen ser compactos y ligeros, lo que permite llevarlos consigo fácilmente.		
		\subsection{Caracteristicas de los dispositivos moviles}
		
		Los dispositivos móviles son herramientas versátiles que permiten una amplia gama de funciones. Facilitan la comunicación a través de llamadas telefónicas, mensajes de texto, correos electrónicos y aplicaciones de mensajería instantánea, así como el acceso a Internet para navegar por sitios web, redes sociales y servicios en la nube. 
		
		Además, ofrecen funciones multimedia como tomar fotos, grabar videos y reproducir música y vídeo. Con la capacidad de instalar y ejecutar una variedad de aplicaciones, los dispositivos móviles son útiles para actividades de productividad, entretenimiento, educación y salud. 
		
		También están equipados con una variedad de sensores, como GPS, acelerómetros y giroscopios, que permiten funciones adicionales como navegación, detección de movimiento y realidad aumentada.
		
		\subsection{Componentes de los dispositivos moviles}
		En cuanto a los componentes de los dispositivos móviles, pueden variar según el dispositivo, de manera general incluyen:
		
		• Procesador (CPU): Responsable de ejecutar las instrucciones y procesar datos.
		
		• Memoria RAM: Almacena temporalmente datos y programas en ejecución.
		
		• Almacenamiento interno: Donde se guardan permanentemente los datos y las aplicaciones.
	
		• Pantalla táctil: Permite la interacción con el dispositivo a través de gestos táctiles.
	
		• Batería: Suministra energía para el funcionamiento del dispositivo.
		
		• Cámaras: Para capturar fotos y videos.
		
		• Sensores: Como el GPS, acelerómetro, giroscopio, etc., que permiten diversas funciones como la navegación, detección de movimiento, realidad aumentada, entre otros.
		
		• Conectividad Wi-Fi: Permite la conexión a redes inalámbricas para acceder a Internet y otros dispositivos.
		
		Un componente relevante es el giroscopio, sensor mide la velocidad angular o la orientación angular de un dispositivo en el espacio. Funciona detectando los cambios en la orientación del dispositivo y proporcionando información sobre su movimiento rotacional.
		
		El giroscopio se compone de un disco o un rotor que gira libremente en un eje. Cuando el dispositivo se mueve o gira, el giroscopio detecta estos cambios en la dirección del movimiento y los convierte en señales eléctricas que pueden ser interpretadas por el sistema operativo y las aplicaciones del dispositivo.
		
		\section{Relacionando la arquitectura y los dispositivos moviles}
		
		La relación entre un dispositivo móvil y su arquitectura nos permite entender cómo este funciona, asi del cómo se diseñan y se optimizan estos dispositivos. La arquitectura de un dispositivo móvil se refiere a la estructura interna del dispositivo, que incluye tanto hardware como software.
		
		Algunas formas en las que se relaciona un dispositivo movil con su arquitectura son:
		
		1. Hardware: La arquitectura de hardware de un dispositivo móvil incluye componentes físicos. La selección y disposición de estos componentes determinan en gran medida el rendimiento, la eficiencia energética y las capacidades del dispositivo.
		1. Software: La arquitectura de software de un dispositivo móvil se refiere al sistema operativo  así como a las capas de software adicionales y las aplicaciones que se ejecutan en el dispositivo. Esto incluye el kernel del sistema operativo, los controladores de dispositivos, las bibliotecas de software, las interfaces de usuario y las aplicaciones de usuario. 
		2. Interacción entre hardware y software: La arquitectura de un dispositivo móvil también abarca la forma en que el hardware y el software interactúan entre sí. el sistema operativo gestiona los recursos de hardware, cómo las aplicaciones acceden y utilizan estos recursos, y cómo se optimiza el rendimiento y la eficiencia energética mediante la coordinación entre hardware y software.
		3. Evolución y mejoras: La arquitectura de un dispositivo móvil no es estática y evoluciona con el tiempo para adaptarse a las nuevas tecnologías, tendencias y necesidades del mercado. Por lo tanto, la arquitectura debe ser flexible y escalable para permitir actualizaciones de hardware y software, así como mejoras en el rendimiento, la seguridad y la usabilidad del dispositivo.
		
		
		\subsection{Sistemas operativos}
Un sistema operativo (SO) es un software fundamental que actúa como intermediario entre el hardware de una computadora y los programas de aplicación. Es el conjunto de programas y utilidades que gestionan los recursos del sistema y proporcionan servicios a los programas de aplicación. 
Los sistemas operativos para dispositivos móviles gestionan eficientemente los recursos del hardware, como la CPU, la memoria RAM y el almacenamiento, así como los dispositivos de entrada/salida como la pantalla táctil y la cámara, asignando y liberando recursos según sea necesario para garantizar un rendimiento óptimo. Proporcionan una interfaz de usuario intuitiva, ya sea a través de una pantalla táctil con iconos y gestos o mediante comandos de voz, permitiendo a los usuarios interactuar fácilmente con sus dispositivos. 
		
		\subsection{Caracteristicas de los sistemas operativos para los moviles}
		Los sistemas operativos para dispositivos móviles tienen características distintivas que los hacen únicos y se adaptan a diferentes necesidades y preferencias de los usuarios. 
		1. Android:
		◦ Android es conocido por su amplia variedad de dispositivos y su personalización. Ofrece una amplia gama de aplicaciones a través de Google Play Store y permite una personalización profunda del sistema operativo.
		◦ Uso particular: Android es popular entre una amplia gama de usuarios debido a su variedad de dispositivos, precios y personalización. Es utilizado por fabricantes como Samsung, Huawei, Xiaomi y otros.
		2. iOS (Apple):
		◦ Es simple y presenta consistencia en la experiencia del usuario. Ofrece un ecosistema integrado con dispositivos Apple, como iPhone, iPad, Apple Watch y Mac, lo que permite una experiencia fluida entre dispositivos.
		◦ Uso particular: iOS es popular entre los usuarios que prefieren la simplicidad y la integración entre dispositivos. Es comúnmente utilizado por aquellos que son parte del ecosistema Apple y valoran la calidad del hardware y el software.
		3. Windows Mobile:
		◦ Ofrece una integración con Windows en computadoras y una interfaz de usuario familiar basada en azulejos. Tiene un enfoque en la productividad con aplicaciones de Office y servicios de Microsoft.
		◦ Uso particular: Aunque ha perdido relevancia en los últimos años, Windows Mobile es utilizado por algunos usuarios empresariales que valoran la integración con el ecosistema Windows y las herramientas de productividad de Microsoft.
		4. Otros sistemas operativos:
		◦ Existen otros como KaiOS, utilizado en dispositivos básicos y teléfonos inteligentes de gama baja, y sistemas operativos basados en Linux como Ubuntu Touch y Sailfish OS, que ofrecen una alternativa para usuarios más técnicos y entusiastas de la privacidad.
		
		
		\section{Conclusion}
		la arquitectura de los dispositivos móviles abarca tanto el hardware como el software que conforman estos dispositivos portátiles. Desde el procesador y la memoria hasta el sistema operativo y las aplicaciones, cada componente contribuye a la funcionalidad y el rendimiento del dispositivo. 
		
		La combinación de sistema operativo y componentes de hardware influye en la experiencia del usuario y determina las capacidades y el rendimiento del dispositivo móvil.
		
	\end{multicols} 
	\section{References}
	Vujevic, T. (2023, march). "The mobile app architecture guide". Decode Agency. https://decode.agency/article/mobile-app-architecture/
	
	
	
	Shah, M. (2023, October 18th). "Mobile App Architecture: Everything You Need to Know". Radix.
	https://radixweb.com/blog/guide-to-mobile-app-architecture
	
	
	
	Walburg, M. (2021, January 12th). "Mobile architecture – what are the types?". Binarapps
	https://binarapps.com/mobile-architecture-what-are-the-types/
	
\end{document} 
